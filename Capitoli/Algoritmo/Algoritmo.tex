In questo capitolo saranno mostrati i passi da effettuare per ottenere, partendo da un dataset rappresentante una relazione, una lista di dipendenze funzionali rilassate.
La sequenza di passi che l'algoritmo affronterà sono:
\begin{itemize}
	\item \textbf{\emph{Feasibility}}
	\item \textbf{\emph{Minimality}}
	\item \textbf{\emph{Generation}}
\end{itemize}
Per fare in modo che la prima fase(\emph{Feasibility}) abbia inizio, ci dobbiamo creare la matrice delle distanze, che, insieme ad alcune informazioni aggiuntive verranno date in input alla suddetta fase.
In questo capitolo verranno descritte nello specifico le fasi di creazione della \emph{matrice delle distanze} e la fase di \emph{Feasibility} che sono oggetto di questo lavoro di tesi.
\import{Capitoli/Algoritmo/Sezioni/}{MatriceDelleDistanze}
\import{Capitoli/Algoritmo/Sezioni/}{Feasibility}
\import{Capitoli/Algoritmo/Sezioni/}{LastStep}