\section{Minimality ed RFD Generation}
Le fasi di \emph{Minimality e RFD Generation} sono quelle finali dell'algoritmo. La prima fase che beneficerà dei nostri \emph{insiemiC} è quella di minimality. Successivamente si procederà verso la generazione delle \textbf{RFD} trovate mediante la fase di \emph{RFD Generation}. Entrambe le fasi finali non saranno oggetto di studio per questo lavoro di tesi.