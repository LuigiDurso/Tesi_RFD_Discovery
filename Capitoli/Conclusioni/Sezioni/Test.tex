\section{Test}
Tutti i test sono stati eseguiti su una macchina con sistema operativo windows 10, un processore Intel Core i7 4750HQ a 2.0GHz e con 12Gb di RAM DDR3.\\
Per ogni dataset utilizzato abbiamo testato il nostro algoritmo tramite la classe \textbf{MainClass.java} descritto nel capitolo di implementazione, ricavando così i tempi impiegati dal nostro algoritmo su ciascun dataset, provando ogni possibile colonna come RHS e le restanti colonne come LHS.  \\
Per ogni dataset abbiamo testato tutte le combinazioni di attributi sull'RHS e sull'LHS ciascuna per 10 volte in modo da avere una stima più accurata dei tempi. Abbiamo diviso il tempo complessivo in più parti, il tempo impiegato per calcolare la matrice delle distanze, il tempo per il Feasibility ed il tempo per gli ultimi due step. Per un testing più approfondito sono stati effettuati dei test con l'utilizzo di un numero differente di thread. Dato l'hardware su cui sono stati testati i dataset è stato possibile effettuare test su un numero di thread che va da 1 a 7.\\
Mostreremo quelli che sono i test ritenuti rilevanti:
\begin{itemize}
	\item Test in sequenziale;
	\item Test con un numero di thread pari a 2;
	\item Test con un numero di core fisici massimi(thread pari a 3);
	\item Test con thread massimi (pari a 7).
\end{itemize}
\subsection{Dataset utilizzati}
Oltre ad una serie di dataset creati appositamente per verificare la correttezza di alcune operazioni, abbiamo prelevato una serie di dataset dal sito dell'Information Systems Group dell'Hasso-Plattner-Institut \cite{metanome}: un un gruppo di ricerca della suddetta Università tedesca che si occupa, tra le altre cose, di progettare algoritmi dedicati alla ricerca delle dipendenze funzionali. Su tale sito, oltre a poter consultare gli algoritmi sviluppati, è possibile accedere a tutti i dataset sui quali tali algoritmi sono stati testati corredati a varie informazioni (i.e. fonte, numeri di attributi, numero di righe, dipendenze funzionali trovate, dipendenze funzionali ordinate trovate ecc).\\
\begin{table}[h]
	\centering
	\begin{tabular}{lllll}
		\multicolumn{1}{c}{\textbf{Nome}} & \multicolumn{1}{c}{\textbf{Attributi}} & \multicolumn{1}{c}{\textbf{Righe}} & \multicolumn{1}{c}{\textbf{Dimensione}} \\
		\hline
		dataset & 4 & 7 & 118 B \\
		Bridges & 13  & 108 & 6 Kb\\
		balance-scale  & 5  & 624 & 7 Kb\\
		echocardiogram  & 13  & 131 & 6,1 Kb\\
	\end{tabular}
	\caption{Dataset utilizzati}
	\label{datasetUtilizzati}
\end{table}
\subsection{Test sequenziale}
\subsection{Test con due thread}
\subsection{Test con tre thread}
\subsection{Test con sette thread}