Esistono svariati metodi per scoprire le RFD data una determinata soglia $\epsilon$, un esempio è il metodo \textit{top-down}.\\
I metodi di discovery \textit{top-down} effettuano una generazione di possibili FD livello per livello e controllano se queste si verificano. L'algoritmo inizia generando un grafo di attributi, con una struttura a lattice, dove vengono considerati tutti i possibili sottoinsiemi di attributi. Dato uno schema relazionale $R = (A_1, A_2, \cdots, A_n)$, il livello 0 del lattice non contiene nessun attributo, il livello 1 contiene tutti i singleton dei singoli attributi dello schema relazionale $R$, il livello due tutte le possibili coppie di attributi in $R$ fino ad arrivare all'ultimo livello, l'n-esimo, che contiene un unico insieme con tutti gli attributi di $R$ al suo interno. Ogni sottoinsieme contenuto nel lattice rappresenta un candidato per una possibile FD.\\
Generato il lattice, l'algoritmo parte dal livello 0 fino ad arrivare all'ultimo, e per ogni livello verifica, per tutti i possibili sottoinsiemi $X \in L_r$\footnote{livello r-esimo}, l'esistenza di possibili dipendenze funzionali. Nello specifico, per ogni attributo $A \in X$ si cerca di verificare se la FD $X \setminus \{A\} \rightarrow A$ vale. Per ridurre il tempo di esecuzione esponenziale, assieme alla verifica avviene una potatura del grafo sfruttando la scoperta di nuove FD.\\
Inoltre negli ultimi anni c'è stata una proliferazione delle RFDs di cui solo alcune di loro erano dotate dell'algoritmo per la scoperta dai dati. Mostriamo adesso alcune di esse \cite{onDiscovery}.


\section{AFD Discovery}
\import{Capitoli/StatoDellArte/Sezioni/}{AFDdiscovery}
\section{MD Discovery}
\import{Capitoli/StatoDellArte/Sezioni/}{MDdiscovery}
\section{DD Discovery}
\import{Capitoli/StatoDellArte/Sezioni/}{DDdiscovery}