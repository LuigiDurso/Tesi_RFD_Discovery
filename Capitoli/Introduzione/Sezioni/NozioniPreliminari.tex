\section{Nozioni Preliminari}
E' necessario, prima di cominciare con lo studio del nostro algoritmo, introdurre alcuni concetti preliminari volti alla comprensione della logica dietro le RFD.

\subsection{Schema di relazione}
Uno schema di relazione è costituito da un simbolo $R$, detto nome della relazione, e da un insieme di attributi $X = \{A_1,A_2,...,A_n\}$, di solito indicato
con $R(X)$. A ciascun attributo $A \in X$ e associato un dominio $dom(A)$.
Uno schema di base di dati è un insieme di schemi di relazione con nomi
diversi:
\\ \\
\centerline{$R = \{ R_1(X_1),R_2(X_2),\ldots,R_n(X_n)\}$.}
\\ \\
Una relazione su uno schema $R(X)$ è un insieme $r$ di tuple su $X$. Per
ogni istanza $r \in R(X)$, per ogni tupla $t \in r$ e per ogni attributo $A \in X$,
$t[A]$ rappresenta la proiezione di $A$ su $t$. In modo analogo, dato un insieme
di attributi $Y \subseteq X$, $t[Y]$ rappresenta la proiezione di $Y$ su $t$.\cite{libroCeri}
\\

\begin{table}[H]
    \centering
    \begin{tabular}{ | l | l | l | l |}
        \hline
        Matricola & Cognome & Nome & Data di nascita\\
        \hline
        123456 & Rossi & Mario & 25/11/1991 \\ 
        567891 & Neri & Anna & 23/04/1992 \\ 
        \hline
    \end{tabular}
    \caption{Esempio di schema di relazione}
    \label{tab:table example}
\end{table}

\subsection{Dipendenze funzionali canoniche}
Una \textit{dipendenza funzionale}, abbreviata in FD, è un vincolo di integrità semantico per il modello relazionale che descrive i legami di tipo funzionale tra gli attributi di una relazione. \\
Data una relazione $r$ su uno schema $R(X)$ e due sottoinsiemi di attributi non vuoti $Y$ e $Z$ di $X$, diremo che esiste su $r$ una dipendenza funzionale tra $Y$ e $Z$, se, per ogni coppia di tuple $t_1$ e $t_2$ di $r$ aventi gli stessi valori sugli attributi $Y$, risulta che $t_1$ e $t_2$ hanno gli stessi valori sugli attributi $Z$:
\begin{eqnarray}
\forall t_1, t_2 \in r, t_1[Y] = t_2[Y] \implies t_1[Z] = t_2[Z]
\end{eqnarray}
Una dipendenza funzionale tra gli attributi $Y$ e $Z$ viene indicata con la notazione $Y \rightarrow Z$ e viene associata ad uno schema.\\
Se l'insieme $Z$ è composto da attributi $A_1, A_2, \ldots, A_k$, allora una relazione soddisfa $Y \rightarrow Z$ se e solo se essa soddisfa tutte le $k$ dipendenze $Y \rightarrow A_1$, $Y \rightarrow A_2$,..., $Y \rightarrow A_k$. Di conseguenza, quando opportuno, possiamo assumere che le dipendenze abbiano la forma $Y \rightarrow A$, con $A$ singolo attributo. \\
Una relazione funzionale è \textit{non banale} se $A$ non compare tra gli attributi di $Y$. \\
Data una chiave $K$ di una relazione $r$, si può facilmente notare che esiste una dipendenza funzionale tra $K$ ed ogni altro attributo dello schema di $r$. Quindi una dipendenza funzionale $Y \rightarrow Z$ su uno schema $R(X)$ degenera nel vincolo di chiave se l'unione di $Y$ e $Z$ è pari a $X$. In tal caso $Y$ è superchiave per lo schema $R(X)$ . \\
Con la notazione $\left\langle R(X), F \right\rangle$ indicheremo uno schema $R(X)$ su cui è definito un insieme di dipendenze funzionali $F$. Un'istanza $r$ di $R(X)$ viene detta \textit{istanza legale} di  $\left\langle R(X), F \right\rangle$ se soddisfa tutte le dipendenze funzionali in $F$. Infine, data una relazione funzionale $Y \rightarrow Z$, se ogni istanza legale $r$ di $\left\langle R(X), F \right\rangle$ soddisfa anche $Y \rightarrow Z$, allora diremo che $F$ \textit{implica logicamente} $Y \rightarrow Z$, indicato come $F \models Y \rightarrow Z$.\\

\subsection{Dipendenze funzionali rilassate}
\subsection{Teoria delle decisioni}
\subsection{Dominanza}